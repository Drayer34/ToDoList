\documentclass{article}

\usepackage[frenchb]{babel}
\usepackage[T1]{fontenc}
\usepackage[utf8]{inputenc}
\usepackage{graphicx}

\title{Manuel d'utilisation de l'application ToDoList}
\author{Anthony Brunel et Antoine Laurent}

\date{\today}

\begin{document}
\maketitle
\newpage
\tableofcontents
\listoffigures
\newpage

\section{Lancement de l'application}
Pour lancer l'application, il suffit de double cliquer sur ToDoList.jar, de compiler les sources ou bien sous Linux via la commande : \verb+java -jar ToDoList.jar+
\newline
\newline
Une fois l'application lancée on arrive sur la fenêtre principale (Figure~\ref{Fenêtre principale}), c'est à partir de cette fenêtre que l'on va pouvoir gérer nos tâches. 
\newline
Si l'on veut quitter l'application il faut cliquer sur la croix rouge ou bien utiliser la combinaison de touche Alt + f4

\begin{figure}[h]
	\centering
	\includegraphics[scale=0.34]{images/MainDIsplay.png}
	\caption{Fenêtre principale}
	\label{Fenêtre principale}
\end{figure}

%\clearpage
\section{Les actions possibles}
Lors de la première utilisation, lorsque vous êtes sur la fenêtre principale plusieurs options s'offre alors à vous:
\newline

\begin{itemize}
	\item Créer une tâche au long cours.
	\item Créer une tâche ponctuelle.
	\item Créer, modifier ou supprimer une catégorie.
	\item Générer le bilan.
	\item Effectuer des tris sur nos tâches.
\end{itemize}

\subsection{Créer une tâche}
Pour créer une tâche il faut appuyer sur le bouton "Nouvelle tâche" dans le menu (Figure~\ref{Taches barre}), ensuite on pourra choisir entre une tâche au long cours ou ponctuelle.

\begin{figure}
	\centering
	\includegraphics[scale=0.8]{images/MenuTache.png}
	\caption{Ajout d'une tâche}
	\label{Taches barre}
\end{figure}

Une fois choisie il ne reste plus qu'à entrer les informations à votre tâche (Figures \ref{Tâche au long cours} et \ref{Tâche Ponctuelle}). Il faut appuyer sur "Valider" pour valider la tâche.

\begin{figure}[!ht]
	\centering
	\begin{minipage}[t]{5cm}
		\centering
		\includegraphics [scale=0.34]{images/NewTAsk1.png}
		\caption{Tâche au long cours}
		\label{Tâche au long cours}
	\end{minipage}
	\begin{minipage}[t]{5cm}
		\centering
		\includegraphics [scale=0.34]{images/NewTAsk2.png}
		\caption {Tâche Ponctuelle}
		\label{Tâche Ponctuelle}
	\end{minipage}
\end{figure}

\textbf{NB :} On peut soit rentrer les dates au clavier soit les choisir avec le calendrier.
\newline
\par
 Lorsque l'on ajoute une tâche, on retourne sur la fenêtre principale, les tâches au long cours apparaîtront bleues et les tâches ponctuelles vertes. Le nombre de jours restant avant la fin de la tâche sera indiqué à côté de celle-ci.
De plus, si une tâche est en retard un petit panneau rouge vous l'indiquera.(Figure \ref{Fenetre principale 2}).

\begin{figure}[!h]
	\centering
	\includegraphics[scale=0.34]{images/CaptureMainDIsplay3.png}
	\caption{Fenetre Principale avec tâches}
	\label{Fenetre principale 2}
\end{figure}

\clearpage
\subsection{Créer Modifier une catégoire}
Pour créer/modifier une catégorie il suffit de cliquer sur "Option" puis "Catégorie" (Figure \ref{barre Opiton}).

\begin{figure}[h]
	\centering
	\includegraphics[scale=0.6]{images/MenuOption.png}
	\caption{Menu Option}
	\label{barre Opiton}
\end{figure}

Ensuite on peut voir toutes les catégories, en modifier, en ajouter ou en supprimer (Figure \ref{modif Opiton}).

\begin{figure}[h]
	\centering
	\includegraphics[scale=0.8]{images/Capture_edit_categorie.jpg}
	\caption{Modification d'une Catégorie}
	\label{modif Opiton}
\end{figure}

%\clearpage
\subsection{Générer le bilan}

Notre application permet aussi de générer le bilan, pour ceci il faut cliquer sur le bouton "Option" puis "bilan" (Figure \ref{barre Opiton}).
Une nouvelle fenêtre s'ouvre où il faut rentrer la date de début et la date de fin de la période sur laquelle on veut générer le bilan puis cliquer sur "Générer" (Figrue \ref{bilan}).

\begin{figure}
	\centering
	\includegraphics[scale=0.4]{images/CaptureDisplayBilan.png}
	\caption{Fenêtre bilan}
	\label{bilan}
\end{figure}

La fenêtre contiendra les tâches comprise dans le bilan, leur statut (les tâches finies sont en noir les autres sont de la couleur de leur type), et le pourcentage de tâches courantes, en retard et finies sur cette période.

\subsection{Les tris}

Il y a trois tris à disposition, le tris simple (tris par deadline), le tris avancé (tris en fonction des échéances intermédiaires) et le tris par importance. Pour les activer il faut cliquer sur "Option" puis sélectioner le tris souhaité (Figure \ref{barre Opiton}).

\end{document}
